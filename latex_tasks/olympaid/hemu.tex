\documentclass{article}
\usepackage{amsmath}
\usepackage{amssymb}
\usepackage{fancyhdr}
\usepackage{amsfonts}
\usepackage{geometry}
\usepackage{graphicx}
\usepackage{titlesec}
\geometry{
  top=1in,
  bottom=1in,
  left=1in,
  right=1in
}

%Header and Footer
\pagestyle{plain}
\renewcommand{\headrulewidth}{0pt} % remove header line

% Section formatting
\titleformat{\section}{\large\bfseries}{\thesection}{1em}{}
\begin{document}
\thispagestyle{fancy}
\fancyhf{} % clear all header and footer fields
\fancyhead[L]{\includegraphics[width=8cm, height = 2cm]{logo.png}}
\fancyhead[R]{Name: Hemanth Thatavarthi \\ Batch: COMET.FWC33 \\ Date: 04 August 2025}
\fancyfoot[C]{\thepage}
% Title
\vspace*{0.1em}
\section*{Tenth International Olympiad, 1968}

\begin{enumerate}
    \item Prove that there is one and only one triangle whose side lengths are consecutive integers, and one of whose angles is twice as large as another.

    \item Find all natural numbers \( x \) such that the product of their digits (in decimal notation) is equal to \( x^2 - 10x - 22 \).

    \item Consider the system of equations
    $
    ax_1^2 + bx_1 + c = x_2 \\
    ax_2^2 + bx_2 + c = x_3 \\
    \vdots \\
    ax_n^2 + bx_n + c = x_1,
    $
    with unknowns \( x_1, x_2, \dots, x_n \), where \( a, b, c \) are real and \( a \ne 0 \). Let \( \Delta = (b - 1)^2 - 4ac \). Prove that for this system:
    \begin{itemize}
        \item[(a)] if \( \Delta < 0 \), there is no solution,
        \item[(b)] if \( \Delta = 0 \), there is exactly one solution,
        \item[(c)] if \( \Delta > 0 \), there is more than one solution.
    \end{itemize}

    \item Prove that in every tetrahedron there is a vertex such that the three edges meeting there have lengths which are the sides of a triangle.

    \item Let \( f \) be a real-valued function defined for all real numbers \( x \) such that, for some positive constant \( a \), the equation
    $
    f(x + a) = \frac{1}{2} + \sqrt{f(x) - [f(x)]^2}
    $
    holds for all \( x \).
    \begin{itemize}
        \item[(a)] Prove that the function \( f \) is periodic (i.e., there exists a positive number \( b \) such that \( f(x + b) = f(x) \) for all \( x \)).
        \item[(b)] For \( a = 1 \), give an example of a non-constant function with the required properties.
    \end{itemize}

    \item For every natural number \( n \), evaluate the sum
    $
    \sum_{k=0}^{\infty} \left\lfloor \frac{n + 2k}{2k + 1} \right\rfloor = \left\lfloor \frac{n + 1}{2} \right\rfloor + \left\lfloor \frac{n + 2}{4} \right\rfloor + \cdots + \left\lfloor \frac{n + 2k}{2k + 1} \right\rfloor + \cdots
    $
    (The symbol \( \lfloor x \rfloor \) denotes the greatest integer not exceeding \( x \).)

\end{enumerate}
\section*{Eleventh International Olympiad, 1969}

\begin{enumerate}

    \item Prove that there are infinitely many natural numbers \( a \) with the following property: the number \( z = n^4 + a \) is not prime for any natural number \( n \).

    \item Let \( a_1, a_2, \dots, a_n \) be real constants, \( x \) a real variable, and
    $
    f(x) = \cos(a_1 + x) + \frac{1}{2} \cos(a_2 + x) + \frac{1}{4} \cos(a_3 + x) + \cdots + \frac{1}{2^{n-1}} \cos(a_n + x).
    $
    Given that \( f(x_1) = f(x_2) = 0 \), prove that \( x_2 - x_1 = m\pi \) for some integer \( m \).

    \item For each value of \( k = 1, 2, 3, 4, 5 \), find necessary and sufficient conditions on the number \( a > 0 \) so that there exists a tetrahedron with \( k \) edges of length \( a \), and the remaining \( 6 - k \) edges of length 1.

    \item A semicircular arc \( \gamma \) is drawn on \( AB \) as diameter. \( C \) is a point on \( \gamma \) other than \( A \) and \( B \), and \( D \) is the foot of the perpendicular from \( C \) to \( AB \). We consider three circles, \( \gamma_1, \gamma_2, \gamma_3 \), all tangent to the line \( AB \). Of these, \( \gamma_1 \) is inscribed in \( \triangle ABC \), while \( \gamma_2 \) and \( \gamma_3 \) are both tangent to \( CD \) and to \( \gamma \), one on each side of \( CD \). Prove that \( \gamma_1, \gamma_2 \), and \( \gamma_3 \) have a second tangent in common.

    \item Given \( n > 4 \) points in the plane such that no three are collinear. Prove that there are at least
    $
    \binom{n - 3}{2}
    $
    convex quadrilaterals whose vertices are four of the given points.

    \item Prove that for all real numbers \( x_1, x_2, y_1, y_2, z_1, z_2 \), with \( x_1 > 0 \), \( x_2 > 0 \), \( x_1y_1 - z_1^2 > 0 \), \( x_2y_2 - z_2^2 > 0 \), the inequality
    $
    \frac{8}{(x_1 + x_2)(y_1 + y_2) - (z_1 + z_2)^2} \leq \frac{1}{x_1y_1 - z_1^2} + \frac{1}{x_2y_2 - z_2^2}
    $
    is satisfied. Give necessary and sufficient conditions for equality.
\section*{Twelfth International Olympiad, 1970}
\end{enumerate}
\begin{enumerate}

    \item Let \( M \) be a point on the side \( AB \) of triangle \( \triangle ABC \). Let \( r_1, r_2, r \) be the radii of the inscribed circles of triangles \( \triangle AMC, \triangle BMC \), and \( \triangle ABC \), respectively. Let \( q_1, q_2, q \) be the radii of the escribed circles of the same triangles that lie in the angle \( \angle ACB \). Prove that
    $
    \frac{r_1}{q_1} \cdot \frac{r_2}{q_2} = \frac{r}{q}.
    $

    \item Let \( a, b, n \) be integers greater than 1, and let \( a \) and \( b \) be the bases of two number systems. Let \( A_n, A_{n-1} \) be numbers in the system with base \( a \), and \( B_n, B_{n-1} \) be numbers in the system with base \( b \), defined as follows:
    $
    A_n = x_nx_{n-1} \cdots x_0, \quad A_{n-1} = x_{n-1}x_{n-2} \cdots x_0, \\
    B_n = x_nx_{n-1} \cdots x_0, \quad B_{n-1} = x_{n-1}x_{n-2} \cdots x_0,
    $
    with \( x_n \ne 0 \), \( x_{n-1} \ne 0 \). Prove that
    $
    \frac{A_{n-1}}{A_n} < \frac{B_{n-1}}{B_n}
    \quad \text{if and only if} \quad a > b.
    $

    \item The real numbers \( a_0, a_1, \dots, a_n, \dots \) satisfy the condition:
    $
    1 = a_0 \le a_1 \le a_2 \le \cdots \le a_n \le \cdots
    $
    Define the numbers \( b_1, b_2, \dots, b_n, \dots \) by
    $
    b_n = \sum_{k=1}^{n} \left( \frac{1 - a_{k-1}/a_k}{\sqrt{a_k}} \right).
    $
    \begin{itemize}
        \item[(a)] Prove that \( 0 \le b_n < 2 \) for all \( n \).
        \item[(b)] Given \( c \) with \( 0 \le c < 2 \), prove that there exist numbers \( a_0, a_1, \dots \) with the above properties such that \( b_n > c \) for large enough \( n \).
    \end{itemize}

    \item Find the set of all positive integers \( n \) such that the set \( \{n, n+1, n+2, n+3, n+4, n+5\} \) can be partitioned into two subsets whose products are equal.

    \item In the tetrahedron \( ABCD \), \( \angle BDC = 90^\circ \). Suppose the foot \( H \) of the perpendicular from \( D \) to the plane \( ABC \) is the intersection of the altitudes of \( \triangle ABC \). Prove that
    $
    (AB + BC + CA)^2 \le 6(AD^2 + BD^2 + CD^2).
    $
    For what tetrahedra does equality hold?

    \item In a plane there are 100 points, no three of which are collinear. Consider all possible triangles having these points as vertices. Prove that no more than 70\% of these triangles are acute-angled.

\end{enumerate}
\end{document}
